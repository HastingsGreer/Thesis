\documentclass{article}
\usepackage{graphicx} % Required for inserting images
\usepackage{amsmath}
\usepackage{amssymb}

\title{Symmetries of the image registration problem and their application to learning based solutions}
\author{Hastings Greer }
\date{April 2024}

\begin{document}

\maketitle

\section{Introduction}

Image registration is a core tool for medical imaging studies. In my thesis, I
seek to improve the performance of learning based approaches on this task
through a three pronged-approach. First, as a foundation, I am developing the
notation for registration algorithms. By writing registration algorithms as
operators, it becomes ergonomic to prove properties of these algorithms at the
blackboard. Second, I have developed and developing software that allows our
production code to share notation and structure with our blackboard results. We
use these tools to exploit symmetries (inverse consistency and spatial
equivariance) that the solution to the registration problem should have in
order to regularize and accelerate training.

\section{Notation}

Careful development of notation is a fundamental component of progress in image
registration, and one that is easily overlooked (and difficult to publish in
conferences!)

\subsection{Pre-existing Notation}
The standard in the field of image registration is to describe loss functions
such as similarity loss and regularization loss with typeset equations, but
represent the registration algorithm itself using a flowchart.
\cite{balakrishnan2019voxelmorph} Follows this pattern exactly, although the
relatively simple internal structure of their approach means that there is
little structure to capture with improved notation. Multistep architectures
such as \cite{shen2019networks, mok2020large} use flowcharts along with prose
to clarify ambiguities. The need for new notation comes into the forefront int
LapIRN \cite{mok2020large} where there is significant Across these papers,
there is the standard that the images are functions from a domain to
intensities $I: \Omega \rightarrow \mathcal{R}$ and the transform is a function
from the image domain to the image domain $\varphi: \Omega \rightarrow \Omega$.

The canonical unsupervised registration loss in this notation is

$$\mathcal{L}_\text{sim} = \text{Sim}(I^\text{moving} \circ  \varphi , I^\text{fixed}) $$

(In about half of works in the literature, $\varphi$ is used to indicate the a map from the moving space to the fixed space, and then the map used in the similarity loss is written $\varphi^{-1}$ to indicate that the map is used to pull back the moving image features. In these works, $'\varphi'$ is never computed, so $\varphi^{-1}$ just becomes a long name for the pullback map. I have found that it is better to always talk about the pullback map from the fixed space to the moving space and denote it with a single Greek letter, as otherwise the equations always have $\square^{-1}$ dotted about in an unhelpful way, especially as writing it on every map slows whiteboard analysis, I want to use the superscript space for other information, and only including the inverse symbol sometimes is tremendously confusing.)
\subsection{Registration algorithms as the primary object of study}

My central concept in notation is to describe registration algorithms $\Phi:
	\left( (\Omega \rightarrow \mathcal{R}) \times (\Omega \rightarrow \mathcal{R})
	\right) \rightarrow (\Omega \rightarrow \Omega) $. \footnote{Cyclemorph
	\cite{cyclemorph} uses a similar notation, refering to the maps from moving to
	fixed and fixed to moving as $\phi_{XY}$ and $\phi_{YX}$, but does not
	elaborate that arbitrary expressions can be substituted for X and Y} These
algorithms take in a pair of images, and return a map. Written with this
notation, the unsupervised registration loss becomes

$$ \mathcal{L}_\text{sim} = \text{Sim}(I^\text{moving} \circ  \Phi[I^\text{moving}, I^\text{fixed}] , I^\text{fixed}) $$

By focusing on the properties of registration algorithms $\Phi$ instead of maps
$\varphi$, it becomes easy to succinctly describe symmetries of these
algorithms, and to regularize how the transform varies as images change,
instead of just how the transform varies in space.

As an example, one of the simplest regularization losses based on a symmetry is
to require that the registration algorithm be symmetric. One of the early works
advocating this approach for image registration is CycleMorph.

\subsection{Parallelism of Notation and Code}

In my notation, the inverse consistency loss can be written succinctly and
unambiguously as

$$ \mathcal{L}_{inv} = ||\Phi[I^A, I^B] \circ \Phi[I^B, I^A] - id||^2_2$$

$\Phi$ is a torch module, and we take seriously that it returns a function, not a tensor- it returns a python function that maps tensors of coordinates to tensors of coordinates. Thus,

\section{Related Work}

\cite{Mok_2020_CVPR, iglesias2023easyreg}

\section{Components of the Thesis already published}

\subsection{ICON}
\begin{abstract}

	Learning maps between data samples is fundamental. Applications range from
	representation learning, image translation and generative modeling, to the
	estimation of spatial deformations. Such maps relate feature vectors, or map
	between feature spaces. Well-behaved maps should be regular, which can be
	imposed explicitly or may emanate from the data itself. We explore what induces
	regularity for spatial transformations, e.g., when computing image
	registrations. Classical optimization-based models compute maps between pairs
	of samples and rely on an appropriate regularizer for well-posedness. Recent
	deep learning approaches have attempted to avoid using such regularizers
	altogether by relying on the sample population instead. We explore if it is
	possible to obtain spatial regularity using an inverse consistency loss only
	and elucidate what explains map regularity in such a context. We find that deep
	networks combined with an inverse consistency loss and randomized off-grid
	interpolation yield well behaved, approximately diffeomorphic, spatial
	transformations. Despite the simplicity of this approach, our experiments
	present compelling evidence, on both synthetic and real data, that regular maps
	can be obtained without carefully tuned explicit regularizers, while achieving
	competitive registration performance.%and only marginal degradation in registration performance.
\end{abstract}
The first publication of the thesis, \cite{greer2021icon}, was focused on a novel approach to regularizing image registration: optimizing for inverse consistency and then getting regularity as a side effect. This approach only had one hyperparameter, which was aspirationally could allow training on a new dataset with less tuning needed than multigaussian regularizers, which have may hyperparameters to manually adjust. This publication introduced the "Registration Network" notation, including explicity denoting the input images to the registration network, and the operators TwoStep and Downsample that allow hierarchical multistep registration networks to be constructed.
\subsection{GradICON}

\begin{abstract}

	We present an approach to learning regular spatial transformations between
	image pairs in the context of medical image registration. Contrary to
	optimization-based registration techniques and many modern learning-based
	methods, we do not directly penalize transformation irregularities but instead
	promote transformation regularity via an inverse consistency penalty. We use a
	neural network to predict a map between a source and a target image as well as
	the map when swapping the source and target images. Different from existing
	approaches, we compose these two resulting maps and regularize deviations of
	the \emph{Jacobian} of this composition from the identity matrix. This
	regularizer -- \texttt{\textit{GradICON}} -- results in much better convergence
	when training registration models compared to promoting inverse consistency of
	the composition of maps directly while retaining the desirable implicit
	regularization effects of the latter. We achieve state-of-the-art registration
	performance on a variety of real-world medical image datasets using a single
	set of hyperparameters and a single non-dataset-specific training protocol.
\end{abstract}

This paper fulfilled the promise of the ICON paper that an inverse consistency
based regularizer could be easier to tune than existing approaches.
\subsection{ConstrICON}
\cite{Greer2023InverseCB}

\begin{abstract}

	Inverse consistency is a desirable property for image registration. We propose
	a simple technique to make a neural registration network inverse consistent by
	construction, as a consequence of its structure, as long as it parameterizes
	its output transform by a Lie group. We extend this technique to multi-step
	neural registration by composing many such networks in a way that preserves
	inverse consistency. This multi-step approach also allows for
	inverse-consistent coarse to fine registration. We evaluate our technique on
	synthetic 2-D data and four 3-D medical image registration tasks and obtain
	excellent registration accuracy while assuring inverse consistency.%, ie coarse to fine, 

\end{abstract}

Here, we begin to see the power of registration network notation for formally
proving properties of multistep registration algorithms.
\section{Ongoing work and plan for the Thesis}

\subsection{Coordinate Attention with Refinement Layers}

\begin{abstract}
	Image registration estimates spatial correspondences between a pair of images. These estimates are typically obtained via numerical optimization or regression by a deep network. A desirable property of such estimators is that a correspondence estimate (e.g., the true oracle correspondence) for an image pair is maintained under deformations of the input images. Formally, the estimator should be equivariant to a desired class of image transformations. In this work, we present careful analyses of the desired equivariance properties in the context of multi-step deep registration networks. Based on these analyses we 1) introduce the notions of $[U,U]$ equivariance (network equivariance to the \emph{same} deformations of the input images) and $[W,U]$ equivariance (where input images can undergo \emph{different} deformations); we 2) show that in a suitable multi-step registration setup it is sufficient for overall $[W,U]$ equivariance if the first step has $[W,U]$ equivariance and all others have $[U,U]$ equivariance; we 3) show that common displacement-predicting networks only exhibit $[U,U]$ equivariance to translations instead of the more powerful $[W,U]$ equivariance; and we 4) show how to achieve multi-step $[W,U]$ equivariance via a coordinate-attention mechanism combined with displacement-predicting refinement layers (CARL). Overall, our approach obtains excellent practical registration performance on several 3D medical image registration tasks and outperforms existing unsupervised approaches for the challenging problem of abdomen registration.
\end{abstract}

This paper continues the theme of calculating properties of multistep
registration networks by manipulating registration network notation.

I have submitted CARL to Neurips 2024 and am awaiting a response. If it is
rejected, I plan to resubmit to CVPR. For the resubmission, I will add in the 2nd step of training from the GradICON paper to see if this boosts performance, and increase the number of evaluation pairs on the Abdomen1k dataset to 200.

The CVPR submission deadline is in November, but I intend to finish these changes by the end of September

\subsection{Equivariance of Iterated registration steps}




\subsection{Atlas Building, Similarity Measure investigation, and Registration Foundation Models}

The final paper that I aim to produce in this thesis is an extension of the sequence uniGradICON and multiGradICON. These two papers, which I helped to coauthor, developed a foundation model based on the GradICON architecture and training procedure, which is general- the same model is used across different data sets. In the HugeGradICON paper I will scale both the dataset diversity and model size- this is of course what you do with a foundation model. Where the "Equivariance of Iterated Registration Steps" Paper leans heavily toward methodological novelty and mathematical tricks, the HugeGradICON paper will be entirely practical, aiming to boost the performance of the approach on the kinds of cases that it is being used on in the wild. We have a unique opportunity to learn about its real world useage from github issues:

Users will use the registration tool on non-brain-extracted images. The model needs to either enforce brain extraction for brain registration, as is done in EasyReg \cite{easyReg}, or perform adequadely when brain extraction is not performed. We will augment the training set with non-brain-extracted images, and test performance without brain extraction, and add an open source brain extraction model to the CLI if performance without it is not adequate.

In addition, I will use the label image / input image distinction and randomization strategy introduced in MultiGradICON to attempt to boost registration performance on non-brain-extracted images, by sometimes running the model on the brain extracted image but evaluating the loss on the full head image, and vice versa.

Users do not always have access to original images before resampling. Therefore, our model needs to accurately register images that have black bars introduced by the resampling procedure. To achieve this, I will use the "Label image" / "Input" image distinction again, by only adding black bars to the network input image, but not the image on which the loss is evaluated.

Multimodal registration is necessary. The multiGradICON research has found that adding multimodal registration to the training set does improve performance on multimodal tasks, but harms performance on unimodal tasks. This should be amelioriated by increasing network capacity. I intend to further boost multimodal registration performance by adding SynthMorph shapes and brains datasets to the composite dataset.

ww






\bibliographystyle{plain}
\bibliography{references}

\end{document}
